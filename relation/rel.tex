\documentclass{article}

\usepackage[utf8]{inputenc}
\usepackage{scrextend}
\usepackage{graphicx}
% Margins
\topmargin=-0.45in
\evensidemargin=0in
\oddsidemargin=0in
\textwidth=6.5in
\textheight=9.0in
\headsep=0.25in

\title{Progetto di Automated Reasoning\\2021-2022}

\author{ Christian Londero }
\date{15 Febbraio 2022}

\begin{document}
\maketitle

\section{Definizione del problema}
Si consideri una scacchiera $n$ x $n$ ($n$ è dato in input). Si hanno a disposizione $l$ pezzi a forma di L, $s$ pezzi a forma di quadrato e $r$ pezzi a forma di rettangolo (si veda la figura, la dimensione del rettangolo è 3 x 1, il quadrato è 2 x 2 e il lato lungo della L è 2). $l$, $s$, $r$ sono dati in input. L'obbiettivo è di riempire la scacchiera con i pezzi a disposizione in maniera tale da minimizzare le celle vuote/free. Il requisito aggiuntivo è che $f$ celle (date in input) siano già occupate (quindi vietate).\\
Si veda l'esempio (le celle grigie sono già occupate/vietate).

\end{document}